% Options for packages loaded elsewhere
\PassOptionsToPackage{unicode}{hyperref}
\PassOptionsToPackage{hyphens}{url}
%
\documentclass[
]{article}
\usepackage{amsmath,amssymb}
\usepackage{lmodern}
\usepackage{ifxetex,ifluatex}
\ifnum 0\ifxetex 1\fi\ifluatex 1\fi=0 % if pdftex
  \usepackage[T1]{fontenc}
  \usepackage[utf8]{inputenc}
  \usepackage{textcomp} % provide euro and other symbols
\else % if luatex or xetex
  \usepackage{unicode-math}
  \defaultfontfeatures{Scale=MatchLowercase}
  \defaultfontfeatures[\rmfamily]{Ligatures=TeX,Scale=1}
\fi
% Use upquote if available, for straight quotes in verbatim environments
\IfFileExists{upquote.sty}{\usepackage{upquote}}{}
\IfFileExists{microtype.sty}{% use microtype if available
  \usepackage[]{microtype}
  \UseMicrotypeSet[protrusion]{basicmath} % disable protrusion for tt fonts
}{}
\makeatletter
\@ifundefined{KOMAClassName}{% if non-KOMA class
  \IfFileExists{parskip.sty}{%
    \usepackage{parskip}
  }{% else
    \setlength{\parindent}{0pt}
    \setlength{\parskip}{6pt plus 2pt minus 1pt}}
}{% if KOMA class
  \KOMAoptions{parskip=half}}
\makeatother
\usepackage{xcolor}
\IfFileExists{xurl.sty}{\usepackage{xurl}}{} % add URL line breaks if available
\IfFileExists{bookmark.sty}{\usepackage{bookmark}}{\usepackage{hyperref}}
\hypersetup{
  pdftitle={Entrega 2 - Proyecto de Riesgo ANID},
  pdfauthor={Andrea Araya - Pablo Bellei - Luis Cuello - Gino Benedetti - Gabriela Ossa},
  hidelinks,
  pdfcreator={LaTeX via pandoc}}
\urlstyle{same} % disable monospaced font for URLs
\usepackage[margin=1in]{geometry}
\usepackage{longtable,booktabs,array}
\usepackage{calc} % for calculating minipage widths
% Correct order of tables after \paragraph or \subparagraph
\usepackage{etoolbox}
\makeatletter
\patchcmd\longtable{\par}{\if@noskipsec\mbox{}\fi\par}{}{}
\makeatother
% Allow footnotes in longtable head/foot
\IfFileExists{footnotehyper.sty}{\usepackage{footnotehyper}}{\usepackage{footnote}}
\makesavenoteenv{longtable}
\usepackage{graphicx}
\makeatletter
\def\maxwidth{\ifdim\Gin@nat@width>\linewidth\linewidth\else\Gin@nat@width\fi}
\def\maxheight{\ifdim\Gin@nat@height>\textheight\textheight\else\Gin@nat@height\fi}
\makeatother
% Scale images if necessary, so that they will not overflow the page
% margins by default, and it is still possible to overwrite the defaults
% using explicit options in \includegraphics[width, height, ...]{}
\setkeys{Gin}{width=\maxwidth,height=\maxheight,keepaspectratio}
% Set default figure placement to htbp
\makeatletter
\def\fps@figure{htbp}
\makeatother
\setlength{\emergencystretch}{3em} % prevent overfull lines
\providecommand{\tightlist}{%
  \setlength{\itemsep}{0pt}\setlength{\parskip}{0pt}}
\setcounter{secnumdepth}{-\maxdimen} % remove section numbering
\ifluatex
  \usepackage{selnolig}  % disable illegal ligatures
\fi

\title{Entrega 2 - Proyecto de Riesgo ANID}
\author{Andrea Araya - Pablo Bellei - Luis Cuello - Gino Benedetti -
Gabriela Ossa}
\date{2021-07-09}

\begin{document}
\maketitle

\documentclass{article}

\usepackage[utf8]{inputenc}
\usepackage[spanish]{babel}
\usepackage{amsfonts}
\usepackage{amsmath}
\usepackage{graphicx}
\usepackage{color}
\usepackage[a4paper]{geometry}
\usepackage{float}
\usepackage{longtable}
\usepackage{listings}

\geometry{top=3 cm, bottom=3cm, left=1.75cm, right=1.75cm}

\textbackslash begin\{document\}

\begin{figure}
    \centering
    \includegraphics[scale=0.2]{logoUC.jpg}
\end{figure}

\begin{center}
    \begin{Large}\textbf{Informe de propuesta de proyecto de riesgo para ANID}\end{Large}

\vspace*{0,2 cm}

\textbf{Andrea Araya}, \textbf{Luis Cuello}, \textbf{Gino Benedetti}, \\ \textbf{Gabriela Ossa}, \textbf{Pablo Bellei}\\
Diplomado Data Science\\
Pontificia Universidad Católica de Chile, Facultad de Matemática
\end{center}

\vspace{5mm}

\subsection{Descripción de la problemática}

En el proceso de evaluación de los proyectos se identifican falencias en
la ejecución, es decir, hay proyectos que presentan dificultades para
cumplir con las exigencias académicas mínimas para su adecuado cierre,
tales como: la presentación de un \textit{paper} o manuscrito académico
aceptado, publicado o en prensa, el cumplimiento de las exigencias
éticas a través de un informe de seguimiento ético y/o bioético, la
realización de alguna actividad de divulgación científica, entre otras.

\vspace{3mm}

\noindent Por lo tanto, se vuelve necesario para la Subdirección diseñar
un método que identifique patrones (variables) de no cumplimiento
académico de los proyectos, y así poder crear un clasificador de riesgo,
con la finalidad de alertar de forma temprana a los evaluadores, e
implementar medidas de acompañamiento a los y las investigadoras para
lograr el cumplimiento de las exigencias académicas y así, puedan
finalizar exitosamente su investigación.

\subsection{Preguntas a resolver}

Las principales preguntas a responder serán: ¿Cuáles son los proyectos
que están más propensos a presentar dificultades para culminar con éxito
su investigación?, ¿Es posible asignar una clasificación de riesgo a los
proyectos de acuerdo a ciertos patrones?, ¿Existen variables que ayuden
a predecir el riesgo de no cumplimiento de un proyecto?

\subsection{Hipótesis}

El modelo analítico construido a partir de la información disponible,
podrá predecir el incumplimiento académico de los proyectos a partir de
la clasificación de riesgo asociada a ciertas variables.

\section{Alcance del proyecto}

La información se encuentra almacenada en una base de datos administrada
por la Subdirección de Proyectos de Investigación. Se cuenta con datos
de los proyectos aprobados desde el año 1991 hasta la fecha, lo que se
encuentran desagregados por proyecto de investigación, en los tres
principales fondos de investigación individual: Regular, Postdoctorado e
Iniciación en investigación.

\subsection{Variables}

\begin{table}[h!]

    \centering
    \begin{tabular}{|p{5cm}|p{4.5cm}|p{6cm}|}
         \hline
         Item & Variables & Descripción de la variable \\ \hline
         Identificación del proyecto & Folio & \\
         & Año & \\
         & Duración & \\
         & Instrumento & Fondo de investigación (regular, postdoctorado o iniciación) \\
         Identificación Institución & Nombre & \\
         & Tipo & \\
         & Principal & Institución principal asociada \\
         & Secundaria & Institución secundaria asociada \\
         & Región & \\
         & Comuna & \\
         
         
         Identificación  del investigador principal y asociados & RUT & \\
         & Calidad del Investigador & Relación entre el investigador y el proyecto \\ 
         &  Sexo & \\
         &  Nacionalidad & \\
         &  País & \\
         & Comuna &  \\
         & Institución & \\
         Carta de Adjudicación & Puntaje de proyecto & \\
         & Puntaje de corte & \\
         Estado de proyecto & Fecha de inicio & \\
         & Fecha de término & \\
         & Estado proyecto & Resultado de la postulación \\
         Presupuesto & Presupuesto solicitado & \\
         & Presupuesto asignado & \\
         Bioética & Estado bioética  & Estado de cumplimiento bioético  \\
         & Fecha revisado & \\
         & Fecha actualización & \\  
         Proyecto & Disciplina &  Tipo de disciplina según OCDE\\
         & Área & Área del conocimiento\\
         & Cambio de disciplina & \\
         Grupo de evaluación & Código de grupo & Grupo de evaluadores del informe académico \\
         & Código de área del grupo & \\
         Situaciones especiales & Cambio institución & \\
         &  Modificación académica & \\

         Informe académico & Estado etapa & Resultado de la evaluación al informe \\
         
         \hline
    \end{tabular}
    \label{tab:my_label}
\end{table}

\newpage

\noindent Las variables se encuentran en una base de datos relacional y
se necesita una revisión de esta para poder extraer muestras,
transformarlas y poder analizarlas.

\vspace{3mm}

\noindent A continuación se presenta un modelo relacional de las
variables.

\begin{figure}[H]
    \centering
    \includegraphics[width=0.8\textwidth]{Diagrama Base.png}
    \caption{Modelo relacional de las variables}
    \label{fig:modelo relacional}
\end{figure}

\section{Descripción de metodología}

En la Subdirección de Proyectos de Investigación existe información
respecto de los proyectos adjudicados, la cual se recolecta a través de
los procesos de postulación, evaluación, adjudicación y seguimiento
técnico. A partir de esta información histórica, se buscará identificar
las variables que puedan incidir en el incumplimiento académico de un
proyecto y de esta manera diseñar un modelo de clasificación de riesgo.

\vspace{3mm}

\noindent Esta información se tratará de manera anónima para resguardar
la confidencialidad de los/as investigadores/as, para estos efectos se
considera la transformación de las variables de identificación
originales (RUT, folio) a una nueva variable de identificación para
efectos de procesamiento de la información.

\vspace{3mm}

\noindent Para abordar este proyecto se utilizará el método CRISP-DM
(Cross Industry Standard Process for Data Mining), que desglosa el
proceso en seis fases: 1) comprensión del negocio, 2) comprensión de los
datos, 3) preparación de los datos, 4) modelado, 5) evaluación de los
resultados y 6) puesta en marcha. En las primeras etapas, los esfuerzos
estarán puestos en conocer y comprender los mecanismos de control y
seguimiento de los proyectos adjudicados, para lograr identificar las
variables que pueden ser utilizadas para evaluar el riesgo de
incumplimiento. Una vez identificadas las variables, se procederá a la
preparación de los datos para su posterior análisis y modelación.
Finalmente, se evaluarán los resultados respecto de los objetivos
iniciales y la aplicación de la herramienta por parte de los
profesionales de la Subdirección.

\vspace{3mm}

\noindent Los datos necesarios para desarrollar este proyecto serán
obtenidos directamente desde la Subdirección. Estos se encuentran
almacenados en una base de datos institucional y el periodo disponible
es desde el año 1991 a la fecha. En este momento, se cuenta con la
autorización para llevar a cabo esta iniciativa y utilizar los datos
disponibles.

\begin{figure}[H]
    \centering
    \includegraphics[scale=0.35]{esquema.jpeg}
    \caption{Modelo relacional de la metodología del proyecto}
    \label{fig:my_label}
\end{figure}

\hypertarget{anexo}{%
\subsection{Anexo}\label{anexo}}

\hypertarget{carta-gantt}{%
\subsubsection{Carta Gantt}\label{carta-gantt}}

\includegraphics{CartaGantt.jpeg}

\hypertarget{segundo-reporte}{%
\subsection{SEGUNDO REPORTE}\label{segundo-reporte}}

\hypertarget{comprensiuxf3n-del-negocio}{%
\subsection{1. Comprensión del
negocio}\label{comprensiuxf3n-del-negocio}}

El ciclo de un proyecto que postula a los instrumentos de FONDECYT de la
ANID se grafica en la siguiente imagen. Se inicia con el proceso de
postulación, seguido por la evaluación y posterior adjudicación. Si un
proyecto resulta adjudicado, entonces entra a un proceso de seguimiento,
el cual se estructura por dos aspectos: el aspecto técnico y el aspecto
financiero. En este trabajo nos acotaremos al seguimiento técnico.

\includegraphics{Flujo Gral.jpg}

Parte de la evaluación técnica incluye la elaboración y entrega de un
informe académico, el cual es evaluado por la ANID, y a partir de su
resultado se puede clasificar el proyecto como `aprobado', `rechazado' o
`en proceso'. A continuación se presenta un flujo con los detalles
respecto de la revisión y fallo de los aspectos comprometidos.

\includegraphics{Eval Inf.jpg}

\hypertarget{comprensiuxf3n-de-los-datos}{%
\subsection{2. Comprensión de los
datos}\label{comprensiuxf3n-de-los-datos}}

Los datos están estructurados en nueve tablas, todas disponibles en
formato excel. A continuación, se identifica cada tabla, con la cantidad
de variables y registros que contienen. Previamente, cada tabla fue
innominada a requerimiento de la Institución. En total existen 55
variables únicas. La \emph{primary key} de todas las tablas corresponde
al código que identifica a cada proyecto, denominado
\texttt{cod\_folio}.

\begin{longtable}[]{@{}lll@{}}
\toprule
Nombre tabla & Número de variables & Número de registros \\
\midrule
\endhead
Data general & 25 & 14.850 \\
Etapa & 8 & 47.253 \\
Bioetica & 11 & 6.010 \\
Disciplinas & 13 & 4.093 \\
Miembros & 8 & 3.198 \\
Monto & 5 & 14.874 \\
Situacion especial & 13 & 4.221 \\
Cambios\_a & 9 & 40.741 \\
Cambios\_b & 8 & 40.083 \\
\bottomrule
\end{longtable}

Inicialmente trabajaremos con dos tablas, una que contiene los datos
generales del proyecto denominada \texttt{data\_gral} y otra llamada
\texttt{etapa}, que incluye datos específicos sobre el avance del
proyecto.

\begin{itemize}
\tightlist
\item
  La tabla \texttt{data\_gral} incluye las siguientes variables: año
  proyecto, duración, fecha de término, tipo de concurso, disciplina,
  universidad, entre otras.
\item
  La tabla \texttt{etapa} incluye las siguientes variables: fecha
  actualización del estado de etapa, el estado (aprobado, rechazado o en
  ejecución), entre otras.
\end{itemize}

En la siguiente figura se muestra un modelo relacional para los datos.

\includegraphics{tabla_2.png}

\hypertarget{preparaciuxf3n-de-los-datos}{%
\subsection{3. Preparación de los
datos}\label{preparaciuxf3n-de-los-datos}}

\hypertarget{carga-y-limpieza-de-datos}{%
\subsubsection{3.1 Carga y limpieza de
datos}\label{carga-y-limpieza-de-datos}}

Se utilizaron las siguientes funciones para la carga y limpieza de
datos:

\begin{itemize}
\tightlist
\item
  Carga de datos : \texttt{readxl::read\_excel}
\item
  Visualización formato variables : \texttt{dplyr::glimpse}
\item
  Visualización de datos faltantes : \texttt{naniar::vis\_miss}
\end{itemize}

Los principales problemas detectados en la limpieza y su solución se
presentan a continuación:

\begin{longtable}[]{@{}ll@{}}
\toprule
Problema & Solucion \\
\midrule
\endhead
Variables caracter cargan como númerica & Se aplica función
\texttt{as.character} \\
Nombres campos con espacios & Se aplica función
\texttt{janitor::clean\_names} \\
Entradas con datos faltantes & Se aplica \texttt{na.omit} \\
\bottomrule
\end{longtable}

\hypertarget{limpieza-tabla-data-general}{%
\subsubsection{3.1.1 Limpieza tabla Data
General}\label{limpieza-tabla-data-general}}

A continuación se muestra el retorno de las funciones
\texttt{dplyr::glimpse} y \texttt{naniar::vis\_miss} aplicadas a la
tabla \texttt{data\_gral} depués de la limpieza.

\begin{verbatim}
## Rows: 14,850
## Columns: 25
## $ cod_folio            <chr> "1061", "1062", "1063", "1064", "1065", "1066", "~
## $ sexo                 <chr> "M", "M", "F", "F", "M", "M", "M", "M", "M", "M",~
## $ c_t_relac            <chr> "1", "1", "1", "1", "1", "1", "1", "1", "1", "1",~
## $ gl_nacion            <chr> "CHILE", "CHILE", "CHILE", "CHILE", "CHILE", "CHI~
## $ c_region_del_ir      <chr> NA, NA, "13", "12", "13", NA, "13", "13", "13", "~
## $ cod_rut              <chr> "1978", "1910", "1256", "3501", "1614", "1576", "~
## $ gl_est_proy          <chr> "APROBADO", "APROBADO", "APROBADO", "APROBADO", "~
## $ f_ini_proy           <dttm> 2006-03-15, 2006-03-15, 2006-03-15, 2006-03-15, ~
## $ f_fin_proy           <dttm> 2009-03-15, 2008-03-15, 2009-03-15, 2010-03-15, ~
## $ duracion             <dbl> 3, 2, 3, 4, 3, 3, 4, 3, 3, 3, 3, 4, 4, 4, 3, 3, 4~
## $ gl_tiproy            <chr> "REGULAR", "REGULAR", "REGULAR", "REGULAR", "REGU~
## $ agno_concurso        <dbl> 2006, 2006, 2006, 2006, 2006, 2006, 2006, 2006, 2~
## $ titulo               <chr> "ESTUDIO DE LA RESPUESTA A EMBUTICION DE ACEROS D~
## $ grupo_estudoop       <chr> "INGENIERIA 1", "CS. ECONOM/ADMI", "SOCIOLOGIA CS~
## $ disciplina           <chr> "INGENIERIA DE MATERIALES", "FINANZAS", "CAMBIO S~
## $ gr_rel_disc          <chr> "1", "1", "1", "1", "1", "1", "1", "1", "1", "1",~
## $ c_disciplina         <chr> "86", "217", "164", "159", "43", "93", "36", "36"~
## $ sector_aplicacion    <chr> "TECNICAS DE MANUFACTURAS Y DE PROCESOS.", "CONOC~
## $ c_region_ejecucion   <chr> "13", "13", "13", "12", "13", "13", "13", "13", "~
## $ institucion          <chr> "UNIV.DE SANTIAGO DE CHILE", "UNIV.ADOLFO IBANEZ"~
## $ facultad             <chr> "FAC.DE INGENIERIA", "ESCUELA DE NEGOCIOS-SANTIAG~
## $ depto_unidad         <chr> "DEPTO.ING. METALURGICA", "ESCUELA DE NEGOCIOS-SA~
## $ c_region_institucion <chr> "13", "13", "13", "12", "13", "13", "13", "13", "~
## $ f_nacimiento         <dttm> 1960-02-01, 1972-09-08, 1946-11-26, 1972-09-08, ~
## $ miles_de_ppto_asig   <dbl> 32543, 16802, 28100, 124425, 116028, 45715, 84791~
\end{verbatim}

\includegraphics{Repo-2_files/figure-latex/unnamed-chunk-2-1.pdf}

Notar los siguientes aspectos relevantes: a) que en total se
identificaron 14.850 proyectos para el periodo 2006-2021, b) que pueden
ser clasificados por tipo de proyecto \texttt{gl\_tiproy} (iniciación,
postdoctorado y regular), y c) que la duración de cada uno puede variar
entre uno y cuatros años, los cuales denominaremos etapas. \footnote{Recordar
  que la tabla etapa actualiza el estado del proyecto anualmente}

\hypertarget{limpieza-tabla-etapa}{%
\subsubsection{3.1.1 Limpieza tabla Etapa}\label{limpieza-tabla-etapa}}

La tabla \texttt{etapa} incluye tantas entradas con el mismo número de
folio como años de duración del proyecto. Por ejemplo, un mismo proyecto
puede aparecer hasta cuatro veces (duró cuatro años) en dicha tabla, con
dos entradas `en ejecucion', una `rechazado' y otra `aprobado', pero
todas con diferentes fechas de actualización. Es por ello, que primero
se tuvo que eliminar todos excepto el último registro por código de
folio. Y también eliminar las filas con datos faltantes, porque no se
sabe si esos proyectos se repiten dentro de la tabla pero para otro año.

Además, fue necesario reclasificar las categorías que toma la variable
\texttt{gl\_est\_etapa}, dado que tiene 24 categorías y estas no son
informativas por si mismas. De este modo, la nueva variable
\texttt{etapas\_agrupadas} solo toma tres valores \texttt{APROBADO},
\texttt{RECHAZADO} o \texttt{EN\ PROCESO}. Se entenderá por:

\begin{itemize}
\tightlist
\item
  \texttt{APROBADO}: cumple con las exigencias académicas establecidas
  en conformidad a las bases del concurso.
\item
  \texttt{RECHAZADO}: no cumple con las exigencias académicas
  establecidas en conformidad a las bases del concurso, por lo tanto, se
  exigen rectificaciones que puedan cambiar esta situación.
\item
  \texttt{EN\ PROCESO}: aquellos proyectos vigentes que se encuentran en
  ejecución, cuyo informe académico se encuentra recibido o en
  evaluación.
\end{itemize}

Por último, se añade una columna tiempo que equivale a la diferencia de
tiempo límite inicial del proyecto menos el tiempo que tomó el proyecto
en terminar.

A continuación se muestra el retorno de las funciones
\texttt{dplyr::glimpse} y \texttt{naniar::vis\_miss} aplicadas a la
tabla \texttt{etapa} después de la limpieza.

\begin{verbatim}
## Rows: 25,016
## Columns: 10
## $ cod_folio        <chr> "1061", "1061", "1061", "1062", "1062", "1063", "1063~
## $ agno_etapa       <dbl> 2006, 2007, 2008, 2006, 2007, 2006, 2007, 2008, 2006,~
## $ c_t_proyecto     <chr> "1", "1", "1", "1", "1", "1", "1", "1", "1", "1", "1"~
## $ gl_est_etapa     <chr> "APROBADA", "APROBADA", "APROBADA", "APROBADA", "APRO~
## $ f_est_etapa      <dttm> 2007-05-07, 2008-04-11, 2009-04-14, 2007-05-07, 2012~
## $ duracion         <dbl> 3, 3, 3, 2, 2, 3, 3, 3, 4, 4, 4, 4, 3, 3, 3, 3, 3, 3,~
## $ f_fin_proy       <dttm> 2009-03-15, 2009-03-15, 2009-03-15, 2008-03-15, 2008~
## $ gl_grup_est      <chr> "INGENIERIA 1", "INGENIERIA 1", "INGENIERIA 1", "CS. ~
## $ tiempo           <drtn> 678 days, 338 days, -30 days, 313 days, -1453 days, ~
## $ etapas_agrupadas <chr> "APROBADO", "APROBADO", "APROBADO", "APROBADO", "APRO~
\end{verbatim}

\includegraphics{Repo-2_files/figure-latex/unnamed-chunk-3-1.pdf}

Luego de la limpieza, la tabla \texttt{etapa} está lista para ser usada
más adelante.

\hypertarget{anuxe1lsis-exploratorio}{%
\subsubsection{4. Análsis Exploratorio}\label{anuxe1lsis-exploratorio}}

Primero, se grafican las variables de la tabla \texttt{data\_gral} que
se creen importantes, con la finalidad de ver cómo se relacionan entre
sí. Estas son el número total de proyectos por año, desagregados por
tipo de proyecto y duración.

\includegraphics{Repo-2_files/figure-latex/unnamed-chunk-5-1.pdf}

Luego, mirando sólo la tabla \texttt{etapa}, se quiere conocer la
frecuencia según la clasificación estado de proyecto. Esto es:

\begin{longtable}[]{@{}lr@{}}
\toprule
Var1 & Freq \\
\midrule
\endhead
APROBADO & 10189 \\
EN PROCESO & 173 \\
RECHAZADO & 642 \\
\bottomrule
\end{longtable}

En este punto, se ha extraído información relevante sobre ambas tablas
por separado, pero ha llegado el momento de unirlas y responder
preguntas más interesantes. Entonces, se crea una nueva tabla llamada
\texttt{etapas\_gral}, que es el resultado de un \emph{inner\_join}
entre \texttt{data\_gral}y \texttt{etapa}.

\emph{Verificar de supuestos:}

\begin{enumerate}
\def\labelenumi{\arabic{enumi}.}
\tightlist
\item
  Los proyectos en la categoría de rechazado se exceden en la duración
  oficial del proyecto.
\end{enumerate}

Para verificar el supuesto, se realizó una grafica del delta de tiempo
desde el término de la última etapa del proyecto versus la fecha de
termino del proyecto. Tiempo estimado en días. Además, se muestran los
principales estadísticos de resumen.

\includegraphics{Repo-2_files/figure-latex/unnamed-chunk-7-1.pdf}

\begin{longtable}[]{@{}llrrr@{}}
\toprule
gl\_tiproy & etapas\_agrupadas & mean\_tiempo & max\_tiempo &
std\_tiempo \\
\midrule
\endhead
INICIACION & APROBADO & 304.67568 & 722 & 225.61522 \\
INICIACION & EN PROCESO & 46.07143 & 349 & 99.53157 \\
INICIACION & RECHAZADO & 310.07692 & 699 & 151.36893 \\
POSTDOCTORADO & APROBADO & 266.60684 & 861 & 220.01477 \\
POSTDOCTORADO & EN PROCESO & 306.50000 & 334 & 19.77372 \\
POSTDOCTORADO & RECHAZADO & 294.82857 & 671 & 138.42124 \\
REGULAR & APROBADO & 341.77620 & 1087 & 169.83027 \\
REGULAR & EN PROCESO & 286.83784 & 669 & 117.39878 \\
REGULAR & RECHAZADO & 305.00000 & 638 & 125.81345 \\
\bottomrule
\end{longtable}

Este supuesto no muestra una clara relación entre exceso en el tiempo de
fin de proyecto, dado que muchos proyectos aprobados exceden por mucho
la fecha límite.

\begin{enumerate}
\def\labelenumi{\arabic{enumi}.}
\setcounter{enumi}{1}
\tightlist
\item
  Las macrozona de Chile influye en la posibilidad de caer en
  incumplimiento.
\end{enumerate}

Para verificar la hipótesis, se categorizaron las regiones de ejecución
del proyecto en cuatro zonas. La tabla de análisis contiene 10.404
proyectos y se presenta a continuación.

\begin{verbatim}
##             
##              norte centro centro-sur  sur
##   APROBADO     322   7393       1378 1096
##   EN PROCESO     6    126         23   18
##   RECHAZADO     20    453        100   69
\end{verbatim}

¿Será una función del tiempo?

\begin{verbatim}
##       
##        APROBADO EN PROCESO RECHAZADO
##   2006      0.3        0.0       0.0
##   2007      1.3        0.0       0.0
##   2008      3.9        0.0       0.1
##   2009      4.9        0.0       0.1
##   2010      5.1        0.0       0.1
##   2011      5.1        0.0       0.2
##   2012      6.2        0.0       0.1
##   2013      7.3        0.0       0.1
##   2014      8.5        0.0       0.2
##   2015      9.1        0.0       0.5
##   2016      9.9        0.0       0.6
##   2017      8.9        0.0       0.9
##   2018      8.8        0.1       1.0
##   2019      7.3        0.5       1.0
##   2020      6.0        0.9       0.8
\end{verbatim}

\end{document}
